% Latex template for submission to the 16th International Meeting on Fully 3D Image Reconstruction 
% in Radiology and Nuclear Medicine (Fully3D 2021)
%
% Author: G.Schramm
% Date:   Oct 2020
%
% In case you encouter problems, you can raise a github issue here:
% https://github.com/gschramm/fully3d_2021_templates/issues
%
% 
% To build this document, we recommend to use latexmk via:
% ```latexmk -pdf fully3d_template.tex```
% Building in the online editor overleaf also works.

\documentclass[11pt,twocolumn,twoside]{article}
\usepackage{fully3d}

%%%%%% add your extra packages here (if needed)                                       %%%%%
%%%%%% before, have a look which packages are already imported by the fully3d package %%%%%

\usepackage{amssymb}
\usepackage{algorithmicx}
\usepackage{algpseudocode}

% define float env for algorithm
\usepackage{float}
\floatstyle{ruled}
\newfloat{algorithm}{h}{loa}
\floatname{algorithm}{Algorithm}

%%%%% add your bibtex file that contains the bibtex entries here %%%%%
%%%%% please include DOIs in the bibtex entries if possible      %%%%%
\addbibresource{fully3d_2021.bib}

% custom definitions
\DeclareMathOperator{\proj}{proj}
\DeclareMathOperator{\prox}{prox}

\begin{document}


%-------------------------------------------------------------------------------------------
%%%%% add your title here %%%%%
\title{Fast and memory-efficient reconstruction of sparse TOF PET data with non-smooth priors} 

%%%%% add authors and affiliations here %%%%%
\author[1]{Georg~Schramm}
\author[2]{Martin~Holler}

\affil[1]{Department of Imaging and Pathology, Division of Nuclear Medicine,
          KU Leven, Belgium}

\affil[2]{Institute for Mathematics and Scientific Computing, 
          University of Graz, Austria}

%%%%% don't change these 2 lines %%%%%
\maketitle
\thispagestyle{fancy}



%-------------------------------------------------------------------------------------------
%%%%% add your summary (abstract) here               %%%%%%
%%%%% use footnotesize for this section              %%%%%%
%%%%% please stick to the customabstract environment %%%%%% 


\begin{customabstract}
Foo bar.
\end{customabstract}


%-------------------------------------------------------------------------------------------
%%%%% main text                                                %%%%%    
%%%%% remove the dummy content and put your own content here   %%%%% 
%%%%% feel free to choose your own section titles              %%%%% 
%%%%% you don't need to put the content in a separate tex file %%%%%

% dummy_content.tex shows how to add sections, figures, tables, formulas, and references
% remove the following line, it just adds dummy content

Foo bar %\cite{Chambolle2011,Chambolle2018,Ehrhardt2019}.

%-----------------------------------------------------------------------------
\begin{algorithm}[t]
\begin{algorithmic}[1]
%\Function{direct\_tof\_pet}{$d_0,c_0$}

\State \textbf{Initialize} $x,y$, $(S_i)_i,T,(p_i)_i$,
\State $\overline{z} = z = A^T y$
\Repeat
	\State $x = \proj_{\geq 0} (x - T \overline{z})$
	\State Select $i \in \{ 1,\ldots,m+1\} $ randomly according to $(p_i)_i$
	\State \quad $y_i^+ \gets \prox_{D_i^*}^{S_i} ( y_i + S_i  ( A_i x + s_i))$
	\State \quad $\delta z \gets A_i^T (y_i^+ - y_i)$
	\State \quad $y_i \gets y_i^+$
	\State $z \gets z + \delta z$
	\State $\overline{z} \gets  z + (\delta z/p_i)$
\Until{Stopping criterion fulfilled}
\State \Return{$x$}
%\EndFunction
\end{algorithmic}
\caption{SPDHG for PET reconstruction \cite{Ehrhardt2019}}
\label{alg:algorithm_general}
\end{algorithm}
%-----------------------------------------------------------------------------



%-------------------------------------------------------------------------------------------
\printbibliography

\end{document}
