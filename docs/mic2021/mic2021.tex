\documentclass{IEEEtran}
\usepackage{cite}
\usepackage{amsmath,amssymb,amsfonts}
\usepackage{graphicx}
\usepackage{textcomp,nicefrac}

%-----------------------------------------------------------------------------------
% reduce margins
\usepackage[a4paper,top=2cm,bottom=1.5cm,left=1.0cm,right=1.0cm]{geometry}

\usepackage{algorithmicx}
\usepackage{algpseudocode}

% define float env for algorithm
\usepackage{float}
\floatstyle{ruled}
\newfloat{algorithm}{h}{loa}
\floatname{algorithm}{Algorithm}

% custom definitions
\DeclareMathOperator{\proj}{proj}
\DeclareMathOperator{\prox}{prox}
\DeclareMathOperator*{\argmin}{arg\,min}
%-----------------------------------------------------------------------------------

\usepackage[colorinlistoftodos,prependcaption,textsize=footnotesize]{todonotes}

%-----------------------------------------------------------------------------------

\def\BibTeX{{\rm B\kern-.05em{\sc i\kern-.025em b}\kern-.08em
T\kern-.1667em\lower.7ex\hbox{E}\kern-.125emX}}
\markboth{IEEE NSS/MIC conference 2021}
{Schramm \MakeLowercase{\textit{et al.}}: List-mode SPDHG}

\begin{document}
\title{Fast list-mode reconstruction of sparse TOF PET data with non-smooth priors} 
\author{Georg Schramm and Martin Holler
\thanks{G.S. is with the Department of Imaging and Pathology, Division of Nuclear Medicine,
KU Leuven, Belgium, (e-mail: georg.schramm@kuleuven.be).}
\thanks{M.H. is with the Institute of Mathematics and Scientific Computing, 
University of Graz, Austria}
}

\maketitle

\begin{abstract}
In this work, we propose and analyze a list-mode version of the stochastic primal-dual hybrid gradient
(SPDHG) algorithm which (i) substantially reduces its memory requirements and (ii) reduces
computation time when reconstruction sparse TOF PET data using subsets and non-smooth priors. 
To study the behavior of the proposed algorithm in detail, its performance 
is investigated based on simulated 2D TOF data using a brain-like software phantom.

We find that listmode-version of SPDHG converges almost as fast the original version
which will lead to a substantial improvement in the time needed for a reconstruction
of real 3D TOF PET data.
However, a careful choice of the ratio of the primal and dual step sizes, 
depending on the magnitude of the image to be reconstructed, is crucial to obtain fast convergence.
\end{abstract}

\begin{IEEEkeywords}
Positron emission tomography, Reconstruction algorithms
\end{IEEEkeywords}

\section{Introduction}

Recently, Chambolle et al. \cite{Chambolle2018} and  Ehrhardt et al. \cite{Ehrhardt2019} introduced 
the stochastic primal-dual hybrid gradient (SPDHG) algorithm which is a provably convergent algorithm
that allows to solve the PET reconstruction problem including many non-smooth priors with
only a few iterations.
As discussed in Remark 2 of \cite{Ehrhardt2019}, a potential drawback of SPDHG is that it requires
to keep at least one more complete (TOF) sinogram ($y$) in memory. 
%Moreover, if the proposed preconditioning is used, a second complete (TOF) sinogram
%(the sequence of step sizes $(S_i)_{i=1}^n$) needs to be stored.
In general, this is less of a problem for static single-bed non-TOF PET data, where sinogram sizes
are relatively small.
However, for simultaneous multi-bed, dynamic or TOF PET data, the size of complete sinograms
can be become problematic, especially when using GPUs.
E.g., for modern PET TOF scanners with 25\,cm axial FOV and a TOF resolution of ca. 400\,ps, 
a complete unmashed TOF sinogram in single precision for one bed position 
has approximately $4.4\cdot10^9$ data bins, requiring ca. 17\,GB of memory.
Note that the memory required to store a complete TOF sinogram will further 
increase with better TOF resolution.
Due to the large number of data bins and the limitations in injected dose and acquisition time,
modern TOF sinograms are usually very sparse, meaning that in most data bins no data is
acquired.
E.g., for a typical 3\,min-per-bed-position whole-body FDG scan with an injected dose 
of around 200\,MBq acquired 60\,min p.i. on a state-of-the-art TOF PET/MR scanner, 
more than 95\% of the data (TOF sinogram) bins are empty.
For short early frames in dynamic brain scans, this fraction is even higher.
And even for ``high count'' late static 20\,min FDG brain scans with an injected dose of 150\,MBq
acquired 60\,min p.i., still around 70\% of the data bins are empty.
Due to the very sparse nature of the acquired data, list-mode (event-by-event) type
reconstruction algorithms - such as e.g. list-mode OSEM - are computationally more efficient
than algorithms working with binned data (sinograms).
E.g. a high count (1e8 events) and medium count (1e7 events) acquisition, 
a complete forward and back projection for in list-mode is faster by a factor of 2 and 12,
respectively, compared to a sinogram projection for state of the art TOF PET scanner 
with a TOF resolution of ca. 400\,ps.
Hence, the aim of this work is present a modification of the convergent SPDHG algorithm
that allows efficient processing of PET data in list-mode instead of sinograms.

\section{Methods}

In this work, we focus on time-of-flight (TOF) PET reconstruction using TV regularization, 
noting, however, that generalizations to other non-smooth priors as mentioned above are possible 
within the same framework. 
The TV regularized TOF PET reconstruction method requires to solve the optimization problem
%
\begin{equation}
\argmin _{x\geq 0} \sum_j (Px)_j -  d_j \log \left( (Px)_ j + s_j \right) + \beta \, \|\nabla x\|_{1},
\end{equation}
%
where $x$ is the PET image to be reconstructed, $P$ is the (sinogram) TOF forward projector including 
the effects of attenuation and normalization, $d$ are the acquired prompt TOF coincidences 
(the emission sinogram), and $s$ are additive contaminations including random and scattered coincidences.
The operator $\nabla$ is the gradient operator, $\|\nabla u \|_1$ is sum over all entries of the 
pointwise Euclidean norm of $\nabla u$, and $\beta$ is a scalar controlling the level of regularization.

To work with the emission data in list-mode format, we introduce the list-mode forward operator $P_N$
that maps $x$ to ... 
\todo[inline]{Martin: introduce LM fwd operator}. 

Using the list-mode forward operator $P_N$, we propose the list-mode SPDHG (LM-SPDHG) as shown
in Algorithm~\ref{alg:lmspdhg}.
Note that (i) in every update involving the PET data, only a list-mode forward and back projection of
a subset of the list-mode data is needed and (ii) only one additional array with the same length
as the event list need to be kept in memory during the iterations.
In the pre-processing step 2, the event counts $\mu_e$ need to be calculated once for every event in $N$.
If a certain event detected in bin $i$ is present $n$ times in the event list $N$, then $\mu_e = n$.
Note that in step 5, the images $\bar{z}$ and $z$ need to be initialized with the adjoint sinogram
operator applied on the sinogram $y$ which is a sinogram that is one in all bins where no data was
recorded and zero else. The sinogram $y$ does not need to be kept in memory after this step.

To compare the convergence of SPDHG and LM-SPDHG, we performed reconstructions of simulated
TOF PET data from a virtual 2D scanner mimicking the TOF resolution (ca. 400\,ps FWHM) and 
geometry of one ring (direct plane) of the GE SIGNA PET/MR (sinogram dimension: 
357 radial bins, 224 projection angles, 27 TOF bins).
A software brain phantom with a typical gray to white matter contrast of 4:1 was created
based on the brainweb phantom and used to generate simulated data including the effects
of attenuation and flat contamination (scattered) coincidences
with a simulated scatter fraction of 16\%.
Noisy simulated prompt emission TOF sinograms were generated for $10^6$ counts.

The simulated data were reconstructed with SPDHG and LM-SPDHG using 100 iterations, 56 subsets,
and a fixed level regularization $beta = 0.002$ and different step size ratios.
As in \cite{Ehrhardt2019}, convergence was monitored by tracking the relative cost function
\[ c_\text{rel}(x) = (c(x) - c(x^*)) / (c(x^0) - c(x^*)). \]
compared to an approximate minimizer $x^*$,
which was calculated using the deterministic (sinogram) PDHG with 10000 iterations without subsets.


%-----------------------------------------------------------------------------
\begin{algorithm}[t]
\begin{algorithmic}[1]
\footnotesize
\State \textbf{Input} event list $N$
\State \textbf{Calculate} event counts $\mu_e$ for each e in $N$ (see text)
\State \textbf{Split} event list $N$ into $m$ sublists $N_i$
\State \textbf{Initialize} $x,(S_i)_i,T,(p_i)_i,g$
\State \textbf{Preprocessing} $\overline{z} = z = P^T y$ (see text)
\State \textbf{Initialize} $m$ sub lists $l_{N_i}$ with 0s
\Repeat
	\State $x = \proj_{\geq 0} (x - T \overline{z})$
	\State Select $i \in \{1,\ldots,m+1\}$ randomly accord. to $(p_i)_i$
  \If{$i \leq m$}
	  \State $l_{N_i}^+ \gets \prox_{D^*}^{S_i} \left( l_{N_i} + S_i \left(P_{N_i} x + s_{N_i} \right) \right)$
	  \State $\delta z \gets P_{N_i}^T \left(\frac{l_{N_i}^+ - l_{N_i}}{\mu_{N_i}}\right)$
	  \State $l_{N_i} \gets l_{N_i}^+$
  \Else
	  \State $g^+ \gets \prox_{||\cdot||^*}^{S_i} \left( g + S_i \nabla x \right)$
	  \State $\delta z \gets \nabla^T \left(g^+ - g\right)$
	  \State $g \gets g^+$
  \EndIf
	\State $z \gets z + \delta z$
	\State $\overline{z} \gets  z + (\delta z/p_i)$
\Until{stopping criterion fulfilled}
\State \Return{$x$}
%\EndFunction
\end{algorithmic}
\caption{LM-SPDHG for PET reconstruction}
\label{alg:lmspdhg}
\end{algorithm}
%-----------------------------------------------------------------------------

\begin{figure}[t]
\centerline{\includegraphics[width=1.0\columnwidth]{./figs/brain2d_counts_1.0E+06_beta_2.0E-03_niter_10000_100_nsub_56_precond_False.png}}
\caption{foo bar}
\label{fig:recons}
\end{figure}

\begin{figure}[t]
\centerline{\includegraphics[width=0.8\columnwidth]{./figs/brain2d_counts_1.0E+06_beta_2.0E-03_niter_10000_100_nsub_56_precond_False_metrics.pdf}}
\caption{foo bar}
\label{fig:cost}
\end{figure}


%-----------------------------------------------------------------------------
%-----------------------------------------------------------------------------

\begin{thebibliography}{00}
\bibitem{Chambolle2018}A. Chambolle, M. J. Ehrhardt, P. Richtarik, et al. 
``Stochastic primal-dual hybrid gradient algorithm with arbitrary sampling and imaging
applications``. 
\textit{SIAM Journal on Optimization}, vol. 28, no. 4, 2018

\bibitem{Ehrhardt2019} M. J. Ehrhardt, P. Markiewicz, and C. B. Sch\"onlieb. 
``Faster PET reconstruction with non-smooth priors by randomization and preconditioning``. 
\textit{Phys Med Biol}, vol. 64, no. 22, 2019
\end{thebibliography}


\end{document}
