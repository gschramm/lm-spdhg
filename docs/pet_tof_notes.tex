\documentclass{article}
\usepackage{amsthm}
\usepackage{amsmath}
\usepackage{amssymb}
\usepackage{tikz}
%\usepackage{float}
\usepackage{booktabs}
%\usepackage{listings}
%\usepackage{algorithmicx}
%\usepackage{algpseudocode}
\usepackage{setspace} 
\usepackage{enumerate}
\usepackage{geometry}
\usepackage{graphicx}
\usepackage{xcolor}
%\usepackage[utf8]{inputenc}
\usepackage{ifthen}
\usepackage[hidelinks]{hyperref}

\usepackage{float}
\floatstyle{ruled}
\newfloat{algorithm}{h}{loa}
\floatname{algorithm}{Algorithm}
\usepackage{algpseudocode}
\usepackage{tabularx}
\usepackage{booktabs}
\usepackage{array}



\newtheorem{prop}{Proposition}
\newtheorem{rem}[prop]{Remark}
\newtheorem{thm}[prop]{Theorem}
\newtheorem{defn}[prop]{Definition}
\newtheorem{lem}[prop]{Lemma}
\newtheorem{cor}[prop]{Corollary}
%test



%Math operators

%TV-TGV related
\DeclareMathOperator{\TV}{TV}
\DeclareMathOperator{\TD}{TD}
\DeclareMathOperator{\BV}{BV}
\DeclareMathOperator{\TGV}{TGV}
\DeclareMathOperator{\aTGV}{aTGV}
\DeclareMathOperator{\BGV}{BGV}
\DeclareMathOperator{\fold}{f}
\DeclareMathOperator{\foldn}{fold,n}
\DeclareMathOperator*{\argmin}{argmin}

\newcommand{\TGVak}{{\TGV_\alpha ^k}}
\newcommand{\TGVakm}{\TGV_\alpha ^{k^m}}
\newcommand{\TGVatth}{\TGV_\alpha ^{2^3}}
\newcommand{\BGVk}{\BGV^k}
\newcommand{\TGVat}{\TGV_\alpha ^2}
\newcommand{\BGVt}{\BGV^2}
\newcommand{\BGVkm}{\BGV^{k}(\Omega;\mathbb{R}^m)}
\newcommand{\BGVakm}{\BGV^{k} _\alpha (\Omega;\mathbb{R}^m)}
\newcommand{\TGVkn}{\TGV^{k^n}}


\newcommand{\ctc}{\emph{CT-cvx}}
\newcommand{\ctsc}{\emph{CT-scvx}}
\newcommand{\txt}{\emph{TXT}}
\newcommand{\tgv}{\emph{TGV}}
\newcommand{\bmtd}{\emph{BM3D}}
\newcommand{\cl}{\emph{CL}}


%General
\DeclareMathOperator{\BD}{BD}
\DeclareMathOperator{\KL}{KL}
\DeclareMathOperator{\supp}{supp}
\DeclareMathOperator{\range}{Rg}
\DeclareMathOperator{\dive}{div}
%\DeclareMathOperator{\subdif}{\partial}
\newcommand{\subdif}{\partial}
\DeclareMathOperator{\dint}{D_{\text{int}}}
\DeclareMathOperator{\dpoint}{D_{\text{point}}}
\DeclareMathOperator{\domain}{\text{dom}}
\DeclareMathOperator{\sym}{Sym}
\DeclareMathOperator{\tr}{tr}
\DeclareMathOperator{\loc}{loc}
\DeclareMathOperator{\vspan}{span}
\DeclareMathOperator{\epi}{epi}
\DeclareMathOperator{\proj}{proj}
\DeclareMathOperator{\assign}{assign}
\DeclareMathOperator{\shrink}{shrink}
\DeclareMathOperator{\conv}{conv}
\DeclareMathOperator{\BDCT}{BDCT}
\DeclareMathOperator{\IBDCT}{IBDCT}
\DeclareMathOperator{\Ass}{(A)}

\DeclareMathOperator{\Ck}{(C_k)}
\DeclareMathOperator{\id}{I}
\DeclareMathOperator{\sgn}{sgn}
\DeclareMathOperator{\Sgn}{Sgn}
\DeclareMathOperator{\grad}{\nabla}
%\DeclareMathOperator{\wgrad}{D}
\newcommand{\wgrad}{\mathrm{D}}

\DeclareMathOperator{\inner}{int}

\DeclareMathOperator{\prox}{prox}
\DeclareMathOperator{\diag}{diag}
\newcommand{\dkl}{D_{\text{KL}}}
\newcommand{\dklg}{D^\gamma_{\text{KL}}}

\newcommand{\ds}{{d^ *}}


\newcommand{\infcon}{\,\Box\,}

%\newcommand{\reshape}[2][NM,nn]{\text{reshape}(#2,[#1])}
\newcommand{\reshape}[2][NM,nn]{[#2_{(#1)}]}
%Commands

%Spaces
\newcommand{\C}{ \mathcal{C} }
\newcommand{\B}{ \mathcal{B} }
\newcommand{\G}{ \mathcal{G} }
\newcommand{\symkd}{\sym ^k (\mathbb{R}^d) }
\newcommand{\symkdm}{\sym ^k (\mathbb{R}^d)^m }
\newcommand{\Ccinfty}{C^{\infty} _c (\Omega)}
\newcommand{\Cinfty}{C^{\infty} (\Omega)}
\newcommand{\Coinfty}{C^{\infty} (\overline{\Omega})}
\newcommand{\Lp}{L^p (\Omega)}
\newcommand{\Lq}{L^q (\Omega)}
\newcommand{\D}{ \mathcal{D} }
\newcommand{\st}{ \, \left|\right.\, }
\newcommand{\logl}{L\log L(\Sigma)}
\newcommand{\loglp}{L\log L(\Sigma)_+}

\newcommand{\rank}{\text{rank}}

\newcommand{\tp}{\otimes_\pi}
\newcommand{\itp}{\otimes_\mathfrak{i}}
\newcommand{\itn}{\mathfrak{i}}

\newcommand{\htp}{{\otimes_\sigma}}



%\newcommand{\ctp}{\hat{\otimes}_\pi}
\newcommand{\ctp}{\otimes_\pi}






%Shortcut for vector
\makeatletter
\newcommand{\vc}[2][r]{%
  \gdef\@VORNE{1}
  \left(\hskip-\arraycolsep%
  %\bigg(\hskip-\arraycolsep%
    \begin{array}{#1}\vekSp@lten{#2}\end{array}%
  \hskip-\arraycolsep\right)}
  %\hskip-\arraycolsep\bigg)}

\def\vekSp@lten#1{\xvekSp@lten#1;vekL@stLine;}
\def\vekL@stLine{vekL@stLine}
\def\xvekSp@lten#1;{\def\temp{#1}%
  \ifx\temp\vekL@stLine
  \else
    \ifnum\@VORNE=1\gdef\@VORNE{0}
    \else\@arraycr\fi%
    #1%
    \expandafter\xvekSp@lten
  \fi}
\makeatother

%Usage:
%\vc{1;-2} \quad \vc[l]{1;-2;3}\quad \vc[c]{1;-2;-3}\quad\vc{1;2;-3;4}


\newcommand{\K}{\mathcal{K}}
\newcommand{\Nuc}{\mathcal{N}}
\newcommand{\Lin}{\mathcal{L}}
\newcommand{\HS}{\mathcal{HS}}
\newcommand{\trace}{\text{tr}}
\newcommand{\nuc}{\text{nuc}}



\newcommand{\ncl}{N_\text{cl,s}}

%Sets
\newcommand*{\bigtimes}{\mathop{\raisebox{-.5ex}{\hbox{\huge{$\times$}}}}}
\newcommand{\uint}{U_{\text{int}}}
\newcommand{\upoint}{U_{\text{point}}}

\newcommand{\R}{\mathbb{R}}
\newcommand{\N}{\mathbb{N}}
\newcommand{\Z}{\mathbb{Z}}
\newcommand{\Q}{\mathbb{Q}}

\newcommand{\EXk}{(\text{EX}_k)}
\newcommand{\Du}{\mathrm{D} u }
\newcommand{\vDu}{\vert \mathrm{D} u \vert}
\newcommand{\symgrad}{\mathcal{E}}
\newcommand{\wrt}{\:\mathrm{d}}
\newcommand{\Wrt}{\mathrm{D}}
\newcommand{\limn}{\underset{n\rightarrow \infty }{\lim}}
\newcommand{\wdiv}{W^q(\dive;\Omega)}
\newcommand{\beq}{\begin{equation}}
\newcommand{\eeq}{\end{equation}}
\newcommand{\linear}{\mathcal{L}}
\newcommand{\I}{\mathcal{I}}
\newcommand{\M}{\mathcal{M}}
\newcommand{\T}{\mathcal{T}}
\newcommand{\Pol}{\mathcal{P}}
\newcommand{\embed}{\hookrightarrow}

\newcommand{\vunew}{u_{+}}
\newcommand{\vvnew}{v_{+}}

\newcommand{\note}[1]{\textcolor{red}{NOTE: {#1}}}
\newcommand{\todo}[1]{\textcolor{red}{TODO: {#1}}}


\title{Some notes on TOF-PET reconstruction}

\author{}
\date{}

\begin{document}

\section{A simplification of the stochastic primal-dual algorithm by Ehrhardt for TOF-PET with many zero measurements}
Our aim is to solve the following optimization problem
\[ \min _{u\geq 0} \beta \|\nabla u\|_1 + \sum_j (Pu)_ j -  (d)_j \log( (Pu)_ j + s_ j ) ,\]
where $d$ denotes the given PET data, $s$ the given correction terms. A saddle-point reformulation of this problem reads as follows:
\[ \min _{u\geq 0} (\nabla u,q) - \I_{\|\cdot \|_\infty \leq \beta} (q) + \sum_j (Pu)_jy_j + s_jy_j - D_j^*(y_j) \]
where $D_j^*$ is the convex conjugate of $D_j(y) = y - d_j \log(y)$. 

To solve this problem, we follow the paper \cite{Ehrhardt2019}: Choose subsets $(B_i)_{i=1}^m$ of the PET data corresponding to partial PET-forward operators $(P_i)_{i=1}^m$. 
To simplify notation, we set $P_{m+1} = \nabla$ and choose some probabilities $(p_i)_{i=1}^{m+1}$ which sum up to one. 
Then, for $\rho<1$ (we expect that $\rho\approx 1$ is best), define, for $i=1,\ldots,m$
\[ S_i = \text{diag}(\frac{\rho}{P_i 1} )\qquad  T_i = \text{diag}(\frac{\rho p_i}{P^T_i 1}) \]
and $S_{m+1} = \frac{\rho}{\|\nabla\|}$, $T_{m+1} = \frac{p_i\rho}{\|\nabla\|}$ where we can estimate 
$\|\nabla \| \leq \sqrt{8}$. Further, choose $T = \min_{i=1,\ldots,m+1} T_i$ pointwise and 
save $(S_i)_i$ and $T$ (which is of dimension of one image and one sinogram 

(we will see later that we don not need to save bins with zero data). A good choice for the $(p_i)_i$, according to \cite{Ehrhardt2019}, is $p_i = 1/(2m)$ for $i=1,\ldots,m$ and $p_{m+1} = 1/2$. The direct application of the algorithm of \cite{Ehrhardt2019} then can be found in Algorithm  \ref{alg:algorithm_general}, where we set $(s)_{m+1} = 0$.  Here, for $i=1,\ldots,m$, 
\[ (\prox_{D_i^*}(y))_j = y_j - \frac{y_j - 1 + \sqrt{ (y_j-1)^2 + 4 (S_i)_j \mu d_j}}{2}\]
and 
\[ (\prox_{D_{m+1}^*}(y) )_j = y_j /\max(\beta,|y_j|) \]
where for the latter $y_j \in \R^d$ (with $d \in \{2,3\}$ the dimension), $|\cdot |$ is the Euclidean norm on $\R^d$ and the division is taken pointwise.

\begin{algorithm}[t]
  \begin{algorithmic}[1]
\onehalfspacing

\Function{direct\_tof\_pet}{$d_0,c_0$}

\State \textbf{Initialize} $u,y$, $(S_i)_i,T,(p_i)_i$, $\overline{z}= z = P^T y$
\Repeat
	\State $u = \proj_{\geq 0} (u - T \overline{z})$
	\State Select $i \in \{ 1,\ldots,m+1\} $ randomly according to $(p_i)_i$
	\State \quad $y_i^+ \gets \prox_{D_i^*} ( y_i + S_i  ( P_i u + s_i))$
	\State \quad $\delta z \gets P_i^T (y_i^+ - y_i)$
	\State \quad $y_i \gets y_i^+$
	\State $z \gets z + \delta z, $
	\State $\overline{z} \gets  z + (1/p_i) \delta z$
\Until{Stopping criterion fulfilled}
\State \Return{$x$)}
\EndFunction
\end{algorithmic}
\caption{Direct algorithm for TOF-PET}\label{alg:algorithm_general}
\end{algorithm}

Now we can observe that, for data points $j$ where $d_j = 0$, $(\prox_{D_i^*}(a))_j = 1$ for $a_j \geq 1$ and $(\prox_{D_i^*}(a))_j = a_j$ otherwise. Also, we see that $ a_j = (y_i + S_i (P_iu + s_i) )_j \geq 1$ provided that $(y_i)_j \geq 1$ since all the other quantities are positive. Hence, if we initialize all bins $y_j = 1$ for which the data is zero, it remains constantly equal to $1$ during all iterations. Looking at the algorithm, we see that this means we can just completely ignore these points during the iterations and reduce the dimension of $y$ and the operators $(P_i)_i$ accordingly. Note that the only point where zero bins enter is in the initialization of $z = \overline{z} = P^Ty$, since here $y = 1$ for zero-bins and arbitrary else. Hence we need to carry out one backprojection of the zero bins (which is the best one can expect...).

\section{List-mode reconstruction}

Let $L^1(\Omega)$ be the space of image data on a domain $\Omega \subset \R^3$ and let $S = \{ (s_i)_{i \in M} \st s_i \geq 0 \}$ be the data space in Sinogram representation. That is, the image space is divided into $|M|$ bins (sections of LORs depending on the time of flight) and $s_i$ is the number of counts detected in bin $i \in M$. We assume that we are given a datum in list-mode representation, that is we are provided with a list $N$ of events $e \in N $ where each event corresponds to a count that was detected in a certain bin $i_e \in M$. Further, after a pre-processing step, we assume that for each event $e \in N$ the number of total counts that was detected in the bin $i_e$ is available and we denote this number by $\mu_e \in \N$. (This means, for example, if there are two events $e_1,e_2$ that correspond to the same bin $i_{e_1} = i_{e_2}$ we have that $\mu_{e_1} = \mu_{e_2} = 2$.) 

Furthermore, we assume, for each $i \in M$, $s_i$ to be a given estimate of scatter and randoms. Also, we assume that we have a list-mode representation of the $(s_i)_i$ which (abusing notation) is given as $s_e = s_{i_e}/\mu_e$. 


From the statistical perspective, the correct data fidelity term for image reconstruction from a list-mode dataset $N$ with sinogram representation $d= (d_i)_{ i \in M}$ is 
\[ u \mapsto \KL_d(\hat{K}u):=  \sum_{i \in M} (\hat{K}u)_i + s_i  - d_i -d_i \log(\frac{(\hat{K}u)_i + s_i}{d_i} ) ,\]
where $\hat{K}:L^1(\Omega) \rightarrow S$ is the X-ray transform as mapping from image to Sinogram space. Now we observe that this data term can be reformulated to include the list-mode representation of the data as follows:

\begin{align*}
\KL_d(Ku) 
& = \sum_{i \in M} \sum_{e \in N: i_e = i} \frac{(Ku + s)_i}{\mu_{e}} - 1 - \log \left(\frac{(Ku + s)_i}{\mu_{e}} \right) + \sum_{i \in M: d_i = 0} (Ku)_i  \\
& =  \sum_{e \in N} \frac{(Ku+s)_{i_e}}{\mu_{e}} - 1 - \log \left(\frac{(Ku+s)_{i_e}}{\mu_{e}} \right)
+ \sum_{i \in M: d_i = 0} (Ku)_i \\ 
& = \sum_{e \in N} (Pu+s)_e - 1 - \log \left((Pu+s)_e \right)
+ (Pu)_0
\end{align*}
where we define the linear operator $P$ mapping from $L^1(\Omega)$ to the extended list-mode data $N \cup \{0\}$ as

\[ (Pu) _e = \begin{cases}
 \frac{(Ku)_{i_e}}{\mu_e} & \text{if } e \in N \\
 \sum_{i: \nexists e \in N: i_e = i} (Ku)_{i_e} & \text{else.}
\end{cases}
\]


Within a variation model (e.g., with TV regularization) we can use the reduced data fidelity
\[ D(Pu):= \sum_{e \in N \cup \{ 0 \}} (Pu)_e - d_e \log ( (Pu+s)_e) \]
where $d_e = 1$ for $e \in N$ and $d_0 = 0$. 

With this notation, the update step 6-8 of Algorithm \ref{alg:algorithm_general} for a subset $N_i \subset N$ only operate on the list-mode entries of this subset. That is, they map the image data forward to these list-mode entries (via $P_i= P_{N_i}$), carry out the prox on these entries, an project the data back to image space via $P_{N_i}^T$. The zero-entries only have to be incorporated once at the initialization step, where we have to initialize all list mode entries in $N \cup \{ 0\}$ with $\mu$ (or any value above $\mu$) and have to map back those entries via the adjoint transform $P^T$.
The latter can be done as follows: Denote by $P_N$ the restriction of $P$ to $N$, i.e., $P_N:L^1(\Omega) \rightarrow N$, and by $P_0$ the mapping $u \mapsto \sum_{i: \nexists e \in N: i_e = i} (Ku)_{i_e}$. Then the adjoint of $P$ is given for $(a,b) \in \R^{|N|} \times \R$  as
\[ P^T(a,b) = P_N^T a + P_0^T b \]
with $P_N^T$ the regular adjoint of the list-mode operator and 
\[ P_0^Tb:= \sum_{i: \nexists e \in N: i_e = i} K_i^T b
\]
where $K_iu = (Ku)_i$ and $K_i^T$ is its adjoint (=transpose).




\bibliography{mh_lit_dat}
\bibliographystyle{abbrv}

\end{document}

