%auto-ignore
\\ \textbf{Objective:} 
Complete time of flight (TOF) sinograms of state-of-the-art TOF PET scanners have a large memory 
footprint.
Currently, they contain ${\sim}4{\cdot}10^9$ data bins which amount to ${\sim}17$\,GB 
in 32\,bit floating point precision.
Moreover, their size will continue to increase with advances in the 
achievable detector TOF resolution and increases in the axial field of view.
Using iterative algorithms to reconstruct such enormous TOF sinograms becomes increasingly
challenging due to the memory requirements and the computation time needed to evaluate the
forward model for every data bin.
This is especially true for more advanced optimization algorithms such as the
stochastic primal-dual hybrid gradient (SPDHG) algorithm which allows for the use of non-smooth priors
for regularization using subsets with guaranteed convergence.
SPDHG requires the storage of additional sinograms in memory, which severely limits
its application to data sets from state-of-the-art TOF PET systems using conventional
computing hardware.

\textbf{Approach:}
Motivated by the generally sparse nature of the TOF sinograms, we propose and analyze a new 
listmode (LM) extension of the SPDHG algorithm  for image reconstruction of sparse data 
following a Poisson distribution.
\added{The new algorithm is evaluated based on realistic 2D and 3D simulationsn, 
and a real dataset acquired on a state-of-the-art TOF PET/CT system.
The performance of the newly proposed LM SPDHG algorithm is compared against
the conventional sinogram SPDHG and the listmode EMTV algorithm.} 

\textbf{Main results:}
We show that the speed of convergence of the proposed 
LM-SPDHG is equivalent the original SPDHG operating on binned data (TOF sinograms).
However, we find that for a TOF PET system with 400\,ps TOF resolution and 25\,cm axial FOV,
the proposed LM-SPDHG reduces the required memory from approximately 56\,GB to
0.7\,GB for a short dynamic frame with $10^7$ prompt coincidences and to 12.4\,GB for a long 
static acquisition with $5\cdot10^8$ prompt coincidences.

\textbf{Significance:}
In contrast to SPDHG, the reduced memory requirements of LM-SPDHG enables 
a pure GPU implementation on state-of-the-art GPUs - avoiding memory transfers
between host and GPU - which will substantially accelerate reconstruction times.
This in turn will allow the application of LM-SPDHG in routine clinical practice where short
reconstruction times are crucial.
