%auto-ignore
Complete time of flight (TOF) sinograms of state-of-the-art TOF PET scanners are huge.
Currently, they contain $~4 \cdot 10^9$ data bins which amounts to ~17\,GB in 32\,bit floating 
point precision.
In the foreseeable future, their size will constantly increase due to advances in the 
achievable detector TOF resolution and increases in the axial field of view.

Using iterative algorithms to recontruct such enormous TOF sinograms becomes more and more
challenging because of the memory requirements and the computation time needed to evaluate the
forward model for every data bin.
This is especially true for more advanced optimization algorithms such as the
stochastic primal-dual hybrid gradient (SPDHG) algorithm that allows to use non-smooth priors
for regularization using subsets with guaranteed convergence.
Unfortunately SPDHG requires storing of additional sinograms in memory, which severly limits
its application to data sets from state-of-the-art TOF PET systesms using conventional
computing hardware.

Fortunately, complete TOF sinograms are not only huge but at the same time also extremly
sparse naturally favoring algorithms that are capable of event-by-event (listmode) data
processing.
Motivated by this, we propose and analyze a listmode (LM) version of the SPDHG algorithm 
for image reconstruction of sparse data following a Poisson distribution.

Based on realistic simulations in 2D and 3D, we show that the speed of convergence of LM-SPDHG 
is the same as the one of the original SPDHG operating on binned data (TOF sinograms).
We find that for a state-of-the-art TOF PET system with 400\,ps TOF resolution and 25\,cm axial FOV,
LM SPDHG reduces the required memory from approximately 56\,GB to
0.7\,GB for a short dynamic frame with 1e7 prompt coincidences and to 12.4\,GB for a long 
static acquisition with 5e8 prompt coincidences.

In contrast to SPDHG, the reduced memory requirements of LM-SPDHG consequently enables 
a pure GPU implementation using state-of-the-art GPUs - avoiding memory transfers
between host and GPU - which will substantially accelerate reconstruction times.
This in turn will allow the application of LM-SPDHG in routine clinical practice where short
reconstruction times are crucial.
